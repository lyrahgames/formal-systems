\documentclass[8pt,fleqn,a4paper,twoside]{article}
\usepackage{extsizes}
\usepackage[utf8]{inputenc}
\usepackage[T1]{fontenc}
\usepackage[ngerman]{babel}
\usepackage[bottom=25mm,left=30mm,right=30mm,bottom=35mm]{geometry}
\usepackage{times}
% \linespread{1.15}
\usepackage{multicol}
\usepackage{titlesec}

\titleformat*{\section}{\bfseries\normalsize}
\titleformat*{\subsection}{\bfseries\normalsize}

\usepackage{packages/utilities}
\usepackage{packages/uniinput}

\begin{document}
  \hrule
  \begin{center}
    \Large \bfseries Logiksysteme: Gliederung
  \end{center}
  \bigskip
  \footnotesize
  \begin{minipage}[c]{0.49\textwidth}
    Markus Pawellek \\
    markuspawellek@gmail.com
  \end{minipage}
  \hfill
  \begin{minipage}[c]{0.49\textwidth}
    \raggedleft
    \today
  \end{minipage}
  \medskip
  \normalsize
  \hrule
  \bigskip

  \begin{multicols}{2}

  \section{Aussagenlogik} % (fold)
  \label{sec:aussagenlogik}

    \subsection{Umgangssprachliche Aussagenlogik} % (fold)
    \label{sub:umgangssprachliche_aussagenlogik}
      \begin{itemize}
        \item einführendes Beispiel
        \item Definition umgangsprachliche Aussage
        \item Verknüpfung umgangsprachlicher Aussagen durch Wahrheitstabellen
        \item umgangsprachliche Negation
        \item umgangsprachliche Konjunktion
        \item umgangsprachliche Disjunktion
        \item umgangsprachliche Implikation
        \item umgangsprachliche Äquivalenz
        \item Beispiel mit Wahrheitstabelle
      \end{itemize}
    % subsection umgangssprachliche_aussagenlogik (end)

    \subsection{Formale Aussagenlogik} % (fold)
    \label{sub:formale_aussagenlogik}
      \begin{itemize}
        \item Einführung
        \item Definition aussagenlogische Formeln
        \item Definition Belegung
        \item Definition Erfüllungsrelation
        \item Definition gültige Formel, erfüllbare Formel, unerfüllbare Formel, Tautologie, Kontradiktion
        \item Definition semantische Folgerung
        \item Lemma: Semantische Folgerung verallgemeinert Erfüllungsrelation
        \item Lemma: Zusammenhang zwischen Erfüllungsrelation und Implikation
        \item Definition: semantische Äquivalenz von Formeln
        \item Lemma: $\lnot α\equiv α\to \perp$
        \item Lemma: $α\lor β \equiv (α\to\perp)\to β$
        \item Lemma: $α\land β \equiv (α\to(β\to\perp))\to\perp$
        \item Lemma: äquivalente Ersetzung von Teilformeln
        \item Definition: adäquate Menge von Verknüpfungszeichen
        \item Lemma: $\set{\perp,\to}{}$ ist adäquat
        \item Lemma: Adäquate Mengen von Verknüpfungszeichen
        \item Bemerkung Beweis-Kalkül
      \end{itemize}
    % subsection formale_aussagenlogik (end)

    \subsection{Tableau-Kalkül} % (fold)
    \label{sub:tableau_kalkül}
      \begin{itemize}
        \item einführendes Beispiel mit Expansionsregeln
        \item Definition: Tableau für Formeln aus Atomen $\perp,\top,\lnot,\land$
        \item Systematischer Aufbau eines Tableau
        \item Fast systematischer Aufbau eines Tableau
        \item Definition: Eigenschaften von Pfaden und Tableau
        \item Beispiele
        \item Expansionsregeln für $\land,\lor,\lnot,\to$
        \item Bemerkung: adäquate Mengen reichen für die Beweisbarkeit
        \item Definition: Tableau für Formeln aus Atomen $\perp,\to$
        \item Lemma: endliche Tableau reichen für Formeln
        \item Definition: Tableau-beweisbar
        \item deterministische und nichtdeterministische algorithmische Umsetzung des Tableau-Aufbaus mit Gültigkeitstest
        \item Analyse des Tableau-Algorithmus
      \end{itemize}
    % subsection tableau_kalkül (end)

    \subsection{Frege-Kalkül} % (fold)
    \label{sub:frege_kalkül}
      \begin{itemize}
        \item Definition: Herleitung von Formeln im Frege-Kalkül
        \item Beispiel
        \item Definition: Frege-beweisbar
        \item Lemma: $β\to β$ ist Frege-beweisbar
        \item Lemma: Schlussregel Transitität
        \item Lemma: $\perp$ Frege-beweist $α$
        \item Satz: Deduktionstheorem
        \item Lemma: Umkehrung drittes Axiom
        \item Lemma: ex falso quod libet
        \item Lemma: Doppelnegation
        \item Lemma: ex falso quod libet (allgemeiner)
        \item Lemma: Herleitungen ersetzen Hypothesen
        \item Lemma: Verallgemeinerung der Transitivität
        \item Lemma: Wichtige Theoreme des Frege-Kalküls
        \item Ähnliche Kalküle
      \end{itemize}
    % subsection frege_kalkül (end)

    \subsection{Vollständigkeitsätze der Kalküle} % (fold)
    \label{sub:vollständigkeitsätze_der_kalküle}
      \begin{itemize}
        \item Bemerkung: Einteilung der Formeln in beweisbare und nicht-beweisbare Formeln. Testen der Gültigkeit beweisbarer Formeln. Korrektheit und Vollständigkeit
        \item Bemerkung: Korrektheit des Frege-Kalküls
        \item Lemma: Axiome des Frege-Kalküls sind gültig
        \item Lemma: Korrektheit des Frege-Kalküls
        \item Bemerkung: Vollständigkeit Tableau-Kalkül
        \item Lemma: Pfad bestimmt Belegung, für die alle Formeln erfüllt sind
        \item Lemma: Vollständigkeit des Tableau-Kalküls
        \item Bemerkung: Umwandlung von Tableau-Beweisen in Frege-Beweise
        \item Lemma: aus $\lnot(β\to γ)$ expandierte Formeln sind überflüssig
        \item Lemma: aus $β\to γ$ expandierte Formeln sind überflüssig
        \item Lemma: Frege-Herleitung von $\perp$ aus widersprüchlichen Tableau
        \item Satz: Aus Tableau-Beweisen können Frege-Beweise gemacht werden.
        \item Lemma: Vollständigkeitslemma des Frege-Kalküls
        \item Satz: Vollständigkeitssatz des Frege-Kalküls
        \item Lemma: Korrektheit des Tableau-Kalküls
        \item Satz: Vollständigkeitssatz des Tableau-Kalküls
        \item Bemerkung: Beweisabhängigkeiten
      \end{itemize}
    % subsection vollständigkeitsätze_der_kalküle (end)

  % section aussagenlogik (end)

  \section{Modale Aussagenlogik} % (fold)
  \label{sec:modale_aussagenlogik}

    \begin{itemize}
      \item Einführendes Beispiel
      \item Bemerkung: Bedeutung modale Aussagenlogik
    \end{itemize}

    \subsection{Grundbegriffe der modalen Aussagenlogik} % (fold)
    \label{sub:grundbegriffe_der_modalen_aussagenlogik}
      \begin{itemize}
        \item Einführendes Beispiel
        \item Definition: modallogische Formeln
        \item Definition: Kripke-Modell
        \item Definition: modallogische Erfüllungsrelation
        \item Beispiele
        \item Formelauswertung mit Dynamic Programming
        \item Definition: gültige Formeln, erfüllbare Formeln
        \item Beispiele
        \item Bemerkung: Intuition erfüllbare Formeln
        \item Konstruktion erfüllender Kripke-Modelle
        \item Definition: äquivalente Formeln
        \item Lemma: Wichtige Äquivalenzen
        \item Lemma: adäquate Verknüpfungszeichen für Modallogik
        \item Definition: Verallgemeinerung der modallogischen Erfüllungsrelation
      \end{itemize}
    % subsection grundbegriffe_der_modalen_aussagenlogik (end)

    \subsection{Modallogisches Tableau-Kalkül} % (fold)
    \label{sub:modallogisches_tableau_kalkül}
      \begin{itemize}
        \item Bemerkung
        \item Definition: Tableau für modallogische Formeln aus Atomen $\perp,\to,\square$
        \item Beispiel
        \item Definition: Eigenschaften von Pfaden und Tableau
        \item Lemma: endliche Tableau reichen
        \item Definition: Tableau-beweisbar
        \item Beispiel
        \item Bemerkung: Beweis des Vollständigkeitslemmas
        \item Lemma: Pfad bestimmt Modell
        \item Lemma: Vollständigkeit des modalen Tableau-Kalküls
      \end{itemize}
    % subsection modallogisches_tableau_kalkül (end)

    \subsection{Modallogisches Frege-Kalkül} % (fold)
    \label{sub:modallogisches_frege_kalkül}
      \begin{itemize}
        \item Definition: $\square$Frege-Kalkül
        \item Lemma: Doppelnegation überspringt modale Operatoren
        \item Bemerkung: ursprüngliches Deduktionstheorem nicht möglich
        \item Satz: modallogisches Deduktionstheorem
        \item Lemma: Verallgemeinerung von K
        \item Lemma: Korrektheitslemma für modallogisches Frege-Kalkül
        \item Bemerkung: Vollständigkeit modallogisches Frege-Kalkül
        \item Lemma: aus $\lnot\square β$ expandierte Formeln sind überflüssig
        \item Lemma: Frege-Herleitung von $\perp$ aus widersprüchlichen Tableaux
        \item Satz: Aus Tableau-Beweisen können Frege-Beweise gemacht werden.
      \end{itemize}
    % subsection modallogisches_frege_kalkül (end)

    \begin{itemize}
      \item Satz: Vollständigkeitssätze für die modale Aussagenlogik
      \item Bemerkung: Beweisabhängigkeiten
    \end{itemize}

  % section modale_aussagenlogik (end)

  \section{Algorithmische Umsetzung des Tableau-Kalküls} % (fold)
  \label{sec:algorithmische_umsetzung_des_tableau_kalküls}

    \begin{itemize}
      \item Bemerkung: nichtdeterministischer Algorithmus, rekursive Maximum-Suche, Idee
      \item Gültigkeitstest gemäß Tableau-Kalkül
      \item Analyse des Gültigkeitstests
      \item nichtdeterministischer Algorithmus
      \item Gültigkeitstest für modallogische Formeln
      \item Struktur der rekursiven Aufrufe
      \item Analyse
      \item Satz: Erfüllbarkeitsproblem der modalen Aussagenlogik ist in PSPACE
      \item Komplexität von Logikproblemen
    \end{itemize}

  % section algorithmische_umsetzung_des_tableau_kalküls (end)

  \section{Andere Modallogiken} % (fold)
  \label{sec:andere_modallogiken}
    \begin{itemize}
      \item Beispiel: muddy children
      \item Modellierung von Wissen
      \item Grapheigenschaften
      \item Grapheigenschaften für Kripke-Modelle
      \item Modallogiken mit bestimmten Grapheigenschaften
      \item Wissenslogiken
      \item Andere Logiken mit Kripke-Semantik
      \item Bemerkung: nicht-intuitionistische Beweise
      \item intuitionistische Logik
      \item Definition: Formeln der intuitionistischen Aussagenlogik
      \item Definition: Semantik der intuitionistischen Aussagenlogik
      \item Lemma: Persistenz gilt für alle Formeln
      \item Definition: Gültigkeit
      \item Lemma: Gesetz des ausgeschlossenen Dritten ist intuitionistisch nicht gültig.
      \item Lemma: Eine de Morgan-Regel ist intuitionistisch nicht gültig.
      \item Satz: Satz von Glivenko
      \item Definition: Gödels Übersetzung intuitionistischer Formeln
      \item Satz: Satz von McKinsey und Tarski
    \end{itemize}
  % section andere_modallogiken (end)

  \section{Temporale Aussagenlogik} % (fold)
  \label{sec:temporale_aussagenlogik}

    \begin{itemize}
      \item Einführendes Beispiel und Bemerkung
    \end{itemize}

    \subsection{Grundbegriffe der Zeitlogik LTL und ihrer Kalküle} % (fold)
    \label{sub:grundbegriffe_der_zeitlogik_ltl_und_ihrer_kalküle}
      \begin{itemize}
        \item Definition: LTL-Formeln
        \item Definition: Pfad-Erfüllungsrelation für LTL-Formeln
        \item Bemerkung: intuitive Vorstellung
        \item Definition: äquivalente LTL-Formeln
        \item Lemma: Wichtige Äquivalenzen in LTL
        \item Lemma: adäquate Mengen $\set{\lnot,\land,X,U}{}$ und $\set{\perp,\to,X,U}{}$
        \item Definition: Erfüllbarkeit und Gültigkeit
        \item Definition: Erfüllungsrelation für LTL-Formeln
        \item Beispiel
      \end{itemize}
    % subsection grundbegriffe_der_zeitlogik_ltl_und_ihrer_kalküle (end)

    \subsection{Tableau-Kalkül für LTL-Formeln} % (fold)
    \label{sub:tableau_kalkül_für_ltl_formeln}
      \begin{itemize}
        \item Definition: Tableau für LTL-Formeln aus Atomen $\lnot,\land,X,U$
        \item Notation
        \item Bemerkung
        \item Definition: erfolgloser Pfad eines LTL-Tableau
        \item Definition: erfolgreicher Pfad eines LTL-Tableau
        \item Definition: Tableau-beweisbare LTL-Formeln
        \item Definition: systematisches LTL-Tableau
        \item Lemma: endliche LTL-Tableaux reichen
        \item Satz: Vollständigkeitssatz für Tableau-Beweisbarkeit
      \end{itemize}
    % subsection tableau_kalkül_für_ltl_formeln (end)

    \subsection{Frege-Kalkül für LTL-Formeln} % (fold)
    \label{sub:frege_kalkül_für_ltl_formeln}
      \begin{itemize}
        \item Definition: LTL-Frege-Kalkül
      \end{itemize}
    % subsection frege_kalkül_für_ltl_formeln (end)

    \subsection{Endliche Automaten und reguläre Sprachen} % (fold)
    \label{sub:endliche_automaten_und_reguläre_sprachen}
      \begin{itemize}
        \item Definition: endlicher Automat
        \item Definition: akzeptierte Wörter eines endlichen Automaten
        \item Satz: Äquivalenz von deterministischen und nicht-deterministischen Automaten
        \item Definition: reguläre Sprachen
        \item Satz: Abgeschlossenheit regulärer Sprachen unter Komplement, Vereinigung, Konkatenation und Sternbildung
        \item Definition: Leerheitsproblem
        \item Definition: Nicht-Leerheitsproblem
        \item Algorithmus: Nicht-Leerheitsproblem
        \item Komplexität der Leerheitsprobleme
        \item Definition: Alphabet, endliche Wörter, ω-Wörter
        \item Definition: Büchi-Automat
        \item Beispiel
        \item Lemma: nicht-deterministische Büchi-Automaten können mehr als deterministische Büchi-Automaten
        \item Definition: ω-reguläre Sprachen
        \item Lemma: Abschlusseigenschaften ω-reguläre Sprachen
        \item Definition: Leerheitsproblem für Büchi-Automaten
        \item Algorithmus: Nicht-Leerheitsproblem
        \item Analyse des Algorithmus
        \item Definition: verallgemeinerter Büchi-Automat
        \item Beispiele
        \item Lemma: Jede ω-Sprache, die von einem verallgemeinerten Büchi-Automaten akzeptiert wird, wird auch von Büchiautomat akzeptiert.
      \end{itemize}
    % subsection endliche_automaten_und_reguläre_sprachen (end)

    \subsection{Das Gültigkeitsproblem für LTL} % (fold)
    \label{sub:das_gültigkeitsproblem_für_ltl}
      \begin{itemize}
        \item Gültigkeitsproblem für LTL
        \item Bemerkung: algorithmische Idee
        \item Definition: φ-ähnliche Belegungsfolgen und Mengenfolgen
        \item Definition: lokal-konsistente Menge für φ
        \item Lemma
        \item Definition: Nachbarkonsistenz
        \item Lemma
        \item Definition: globale Konsistenz
        \item Lemma
        \item Folgerung
        \item Lemma
        \item Satz: Charakterisierung Pfad-Erfüllungsrelation
        \item Bemerkung: Konstruktion eines Büchi-Automaten
        \item Beispiel
        \item Definition: durch φ bestimmter Büchi-Automat $B_φ$
        \item Satz
        \item Sazt: Charakterisierung Gültigkeit
        \item Algorithmus: Komplement des Gültigkeitsproblems für LTL
        \item Analyse des Algorithmus
        \item Satz: Gültigkeitsproblem für LTL ist in PSPACE
      \end{itemize}
    % subsection das_gültigkeitsproblem_für_ltl (end)

    \subsection{Weitere temporale Logiken} % (fold)
    \label{sub:weitere_temporale_logiken}
      \begin{itemize}
        \item Computation-tree-logic
        \item Beispiel
        \item Pfad-Quantoren
        \item temporale Logiken
        \item Komplexitätsresultate
      \end{itemize}
    % subsection weitere_temporale_logiken (end)

  % section temporale_aussagenlogik (end)

  \end{multicols}

\end{document}