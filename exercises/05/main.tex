\documentclass[9pt,fleqn,twoside,a4paper]{article}
\usepackage{extsizes}
\usepackage[a4paper,left=27mm,right=27mm,top=30mm,bottom=30mm]{geometry}
\linespread{1.15}\selectfont
\usepackage[utf8]{inputenc}
\usepackage[T1]{fontenc}
\usepackage[ngerman]{babel}
\usepackage{times}
\usepackage[leqno]{amsmath,mathtools}
\allowdisplaybreaks
\usepackage{amssymb}
\usepackage{dsfont}
\usepackage[mathscr]{euscript}
\usepackage{multicol}
\usepackage[autostyle,german=guillemets]{csquotes}
\usepackage{turnstile}

\usepackage{fancyhdr}
\fancypagestyle{titlestyle}{
  \fancyhf{}
  % \fancyfoot[C]{\footnotesize\bigskip\thepage/\pageref{LastPage}}
  \fancyfoot[C]{\footnotesize\bigskip\thepage}
  \renewcommand{\footrulewidth}{0.5pt}
  \renewcommand{\headrulewidth}{0pt}
}
\fancypagestyle{mainstyle}{
  \fancyhf{}
  \fancyfoot[C]{\footnotesize\bigskip\thepage}
  \fancyhead[LO,RE]{\footnotesize \leftmark \smallskip} %left
  \fancyhead[RO,LE]{\footnotesize \rightmark \smallskip} %right
  \renewcommand{\footrulewidth}{0.5pt}
  \renewcommand{\headrulewidth}{0.5pt}
}
% \pagestyle{mainstyle}

\usepackage{uniinput}
\usepackage{utilities}

\renewcommand{\separate}{\quad}
\renewcommand{\implies}{\quad\Longrightarrow\quad}
\renewcommand{\iff}{\quad\Longleftrightarrow\quad}
\newcommand{\fregeProofable}{\sststile{\mathrm{Fre}}{}{}}
\newcommand{\fregeAxiomI}{\mathrm{F1}}
\newcommand{\fregeAxiomII}{\mathrm{F2}}
\newcommand{\fregeAxiomIII}{\mathrm{F3}}
\newcommand{\mendelsonProofable}{\sststile{\mathrm{Men}}{}{}}
\newcommand{\mendelsonAxiomI}{\mathrm{M1}}
\newcommand{\mendelsonAxiomII}{\mathrm{M2}}
\newcommand{\mendelsonAxiomIII}{\mathrm{M3}}
\newcommand{\modusPonens}{\mathrm{MP}}
\newcommand{\fulfills}{\sdtstile{}{}{}}
\newcommand{\infers}{\ddtstile{}{}{}}
\newcommand{\qedBox}{\hfill\ensuremath{\square}}

\begin{document}
  \pagestyle{mainstyle}
  \thispagestyle{titlestyle}
  \hrule
  \begin{center}
    \huge
    % \bfseries
    \scshape
    Logiksysteme \\ Übungsserie 5: Lösungen
  \end{center}
  \medskip
  \footnotesize
  \begin{minipage}[c]{0.49\textwidth}
    Markus Pawellek \\
    markuspawellek@gmail.com
  \end{minipage}
  \hfill
  \begin{minipage}[c]{0.49\textwidth}
    \raggedleft
    \today
  \end{minipage}
  \medskip
  \normalsize
  \hrule
  \bigskip

  \begin{multicols}{2}

  \noindent
  Im Folgenden seien α, β und γ drei beliebige Formeln.
  Wir definieren zunächst die Axiome des Frege-Kalküls.
  \begin{alignat*}{3}
    &\fregeAxiomI(α,β)  &&\define\ α\to(β\to α) \\
    &\fregeAxiomII(α,β,γ)  &&\define\
    \begin{aligned}[t]
      &(α\to(β\to γ)) \\
      &\to ((α\to β)\to(α\to γ))
    \end{aligned}\\
    &\fregeAxiomIII(α)  &&\define\ \lnot\lnot α\to α
  \end{alignat*}
  Die einzige definierte Schlussregel im Frege-Kalkül ist \enquote{Modus Ponens} und wie folgt definiert.
  \[
    \modusPonens(α,α\to β) \define \frac{α\separate α\to β}{β}
  \]
  Aus der Vorlesung sind zudem einige weitere Frege-Theoreme bekannt, die sich vollkommen analog zu den bereits definierten Axiomen verhalten.
  Für die aufgelisteten Frege-Beweise verwenden wir die im Folgenden definierten Theoreme.
  \begin{alignat*}{3}
    &\mathrm{ID}(α) &&\define\ α\to α \\
    &\mathrm{XX}(α,β) &&\define\ (α\to β) \to ((\lnot α\to β)\to β) \\
    &\mathrm{NN}(α) &&\define\ α\to\lnot\lnot α \\
    &\mathrm{EFQL}(α) &&\define\ \perp\to α
  \end{alignat*}
  Zudem erweist es sich als ausgesprochen praktisch weitere Schlussregeln zu beweisen, die die eigentlichen Beweise verkürzen und zudem auch übersichtlicher und verständlicher gestalten.

  \paragraph{Lemma:}
  Im Frege-Kalkül gilt die folgende Schlussregel.
  \[
    \mathrm{TT}(α,β) \define \frac{α}{β\to α}
  \]
  Beweis:
  \begin{align}
    \tag{1}
      & α
      && \text{Hypothese} \\
    \tag{2}
      & α\to(β\to α)
      && \fregeAxiomI(α,β) \\
    \tag{3}
      & β\to α
      && \modusPonens((1),(2))
  \end{align}
  Damit ist die Aussage gezeigt. \qedBox

  \paragraph{Lemma:}
  Im Frege-Kalkül gilt die folgende Schlussregel.
  \[
    \mathrm{TRANS}(α\to β,β\to γ) \define \frac{α\to β \separate β\to γ}{α\to γ}
  \]
  Beweis:
  \begin{align}
    \tag{1}
      & α\to β
      && \text{Hypothese} \\
    \tag{2}
      & β \to γ
      && \text{Hypothese} \\
    \tag{3}
      & α\to(β\to γ)
      && \mathrm{TT}((2),α) \\
    \tag{4}
      &
        \begin{aligned}[t]
          &(α\to(β\to γ)) \\
          &\to((α\to β)\to(α\to γ))
        \end{aligned}
      && \fregeAxiomII(α,β,γ) \\
    \tag{5}
      & (α\to β)\to(α\to γ)
      && \modusPonens((3),(4)) \\
    \tag{6}
      & α\to γ
      && \modusPonens((1),(5))
  \end{align}
  Damit ist die Aussage gezeigt. \qedBox

  \section*{Aufgabe 19} % (fold)
  \label{sec:aufgabe_19}

    \paragraph{(a):} % (fold)
      Es seien $\mathscr{S}$ eine Formelmenge und β und γ beliebige Formeln.
      Es soll die folgende Aussage betrachtet werden.
      \[
        \mathscr{S}\fregeProofable γ \implies S\fregeProofable β\to γ
      \]
      Die Frege-Beweisbarkeit kann durch den folgenden direkten Beweis gezeigt werden.
      \begin{align}
        \tag{1}
          & \mathscr{S}\fregeProofable γ
          && \text{Hypothese} \\
        \tag{2}
          & \mathscr{S}\fregeProofable β\to γ
          && \mathrm{TT}((1),β)
      \end{align}
    % paragraph paragraph_name (end)
    \paragraph{(b):} % (fold)
      Es seien $\mathscr{S}$ eine Formelmenge und β und γ beliebige Formeln.
      Es soll die folgende Aussage betrachtet werden.
      \[
        \mathscr{S}\fregeProofable\lnot β \implies \mathscr{S}\fregeProofable β\to γ
      \]
      Die Frege-Beweisbarkeit kann durch den folgenden direkten Beweis gezeigt werden.
      \begin{align}
        \tag{1}
          & \mathscr{S}\fregeProofable \lnot β
          && \text{Hypothese} \\
        \tag{2}
          & \mathscr{S}\fregeProofable β\to\perp
          && (1), \lnot β = β\to\perp \\
        \tag{3}
          & \mathscr{S}\fregeProofable \perp\to γ
          && \mathrm{EFQL}(γ) \\
        \tag{4}
          & \mathscr{S}\fregeProofable β\to γ
          && \mathrm{TRANS}((2),(3))
      \end{align}
    % paragraph paragraph_name (end)
    \paragraph{(c):} % (fold)
      Es seien $\mathscr{S}$ eine Formelmenge und β und γ beliebige Formeln.
      Es soll die folgende Aussage betrachtet werden.
      \[
        \mathscr{S}\fregeProofableβ \quad \text{und}\quad \mathscr{S}\fregeProofable \lnot γ \implies \mathscr{S}\fregeProofable \lnot(β\to γ)
      \]
      Die Frege-Beweisbarkeit kann durch den folgenden direkten Beweis gezeigt werden.
      \begin{align}
        \tag{1}
          & \mathscr{S}\fregeProofable β
          && \text{Hypothese} \\
        \tag{2}
          & \mathscr{S}\fregeProofable \lnot γ
          && \text{Hypothese} \\
        \tag{3}
          & \mathscr{S}\fregeProofable γ\to\perp
          && (2),\lnot γ = γ\to\perp \\
        \tag{4}
          & \mathscr{S}\fregeProofable β\to(γ\to\perp)
          && \mathrm{TT}((3),β) \\
        \tag{5}
          & \mathscr{S}\fregeProofable
            \begin{aligned}[t]
              &(β\to(γ\to\perp)) \\
              &\to(
                \begin{aligned}[t]
                  &(β\toγ) \\
                  &\to(β\to\perp))
                \end{aligned}
            \end{aligned}
          && \fregeAxiomII(β,γ,\perp) \\
        \tag{6}
          & \mathscr{S}\fregeProofable (β\to γ)\to(β\to\perp)
          && \modusPonens((4),(5)) \\
        \tag{7}
          & \mathscr{S}\fregeProofable β\to((β\to\perp)\to\perp)
          && \mathrm{NN}(β) \\
        \tag{8}
          & \mathscr{S}\fregeProofable (β\to\perp)\to\perp
          && \modusPonens((1),(7)) \\
        \tag{9}
          & \mathscr{S}\fregeProofable (β\to γ)\to\perp
          && \mathrm{TRANS}((6),(8)) \\
        \tag{10}
          & \mathscr{S}\fregeProofable \lnot(β\to γ)
          && (9)
      \end{align}
    % paragraph paragraph_name (end)
    \paragraph{(d):} % (fold)
      Seien nun α eine beliebige Formel und $\mathscr{A}$ eine beliebige Belegung.
      Dann definieren wir $\mathscr{A}_α$ durch die folgende Menge.
      \[
        \mathscr{A}_α\define \mathscr{A}\cup\set{\lnot A_i}{A_i\not\in \mathscr{A} \text{ und $A_i$ kommt in α vor}}
      \]
      Durch die vollständige Induktion sollen die beiden folgenden Aussagen für beliebige Formeln α und beliebige Belegungen $\mathscr{A}$ gezeigt werden.
      \[
        \mathscr{A}\fulfills α \implies \mathscr{A}_α \fregeProofable α
      \]
      \[
        \mathscr{A}\not\fulfills α \implies \mathscr{A}_α \fregeProofable \lnot α
      \]
      Bei $\set{\perp,\to}{}$ handelt es sich um eine adäquate Menge.
      Es reicht also diese Menge von Verknüpfungszeichen in der eigentlichen Induktion zu behandeln.
      \paragraph{Induktionsanfang:} % (fold)
      \label{par:induktionsanfang_}
        % Es sei α nun ein Atom oder Falsum.
        Im Folgenden beschreibe α eine atomare Formel oder $\perp$.
        \paragraph{Fall $α=\perp$:} % (fold)
          Dann wissen wir, dass $\mathscr{A}\not\fulfills α$ gilt.
          Es folgt damit unmittelbar aus dem Prinzip \enquote{ex falsum quod libet} die folgende Aussage, da es sich bei $\mathscr{A}\fulfills α$ um eine falsche Aussage handelt.
          \[
            \mathscr{A}\fulfills α \implies \mathscr{A}_α\fregeProofable α
          \]
          Aus dem folgenden direkten Frege-Beweis lässt sich nun die Aussage $\mathscr{A}_α\fregeProofable\lnot\perp$ schlussfolgern.
          \begin{align}
            \tag{1}
              & \perp \to \perp
              && \mathrm{ID}(\perp) \\
            \tag{2}
              & \lnot\perp
              && (1)
          \end{align}
          Demzufolge muss dann aber auch das Folgende gelten.
          \[
            \mathscr{A}\not\fulfills α \implies \mathscr{A}_α \fregeProofable \lnot α
          \]
        % paragraph fall_ (end)
        \paragraph{Fall $α=A_i$ für $i\in\setNatural$:} % (fold)
          Die folgenden Implikationsketten zeigen die gewünschten Aussagen für atomare Formeln.
          \begin{align*}
            \mathscr{A}\fulfills α &\implies α\in\mathscr{A} \implies α\in\mathscr{A}_α \\
            &\implies \mathscr{A}_α\fregeProofable α
          \end{align*}
          \begin{align*}
            \mathscr{A}\not\fulfills α &\implies α\not\in\mathscr{A} \implies \lnot α \in \mathscr{A}_α \\
            &\implies \mathscr{A}_α \fregeProofable \lnot α
          \end{align*}
        % paragraph fall_ (end)

        Damit gelten die zu beweisenden Aussagen für alle atomare Formeln und $\perp.$
      % paragraph induktionsanfang_ (end)
      \paragraph{Induktionsvoraussetzung:} % (fold)
      \label{par:induktionsvoraussetzung_}
        Es seien α und β beliebige Formeln, für die die folgenden Aussagen gelten.
        \begin{align*}
          \mathscr{A}\fulfills α &\implies \mathscr{A}_α \fregeProofable α \\
          \mathscr{A}\not\fulfills α &\implies \mathscr{A}_α \fregeProofable \lnot α \\
          \mathscr{A}\fulfills β &\implies \mathscr{A}_β \fregeProofable β \\
          \mathscr{A}\not\fulfills β &\implies \mathscr{A}_β \fregeProofable \lnot β \\
        \end{align*}
      % paragraph induktionsvoraussetzung_ (end)
      \paragraph{Induktionsschluss:} % (fold)
        Es seien φ und ϑ beliebige Formeln.
        Dann muss nach Definition das Folgende gelten, da in der Formel $φ\to ϑ$ mindestens genauso viele atomare Formeln vorkommen müssen, wie in φ oder ϑ.
        \begin{align*}
          \mathscr{A}_φ\subset \mathscr{A}_{φ\to ϑ}
          ,\qquad
          \mathscr{A}_ϑ\subset\mathscr{A}_{φ\to ϑ}
        \end{align*}
        Werden zur Hypothesenmenge mehr Formeln hinzugefügt, so lassen sich auch mehr Formeln auf der Basis der Hypothesenmenge beweisen.
        \begin{align*}
          \mathscr{A}_φ \fregeProofable φ \implies \mathscr{A}_{φ\to ϑ} \fregeProofable φ \\
          \mathscr{A}_ϑ \fregeProofable ϑ \implies \mathscr{A}_{φ\to ϑ} \fregeProofable ϑ
        \end{align*}
        Diese Aussagen verwenden wir nun für den eigentlichen Induktionsschluss.
        Aufgrund der adäquaten Menge von Verknüpfungszeichen reicht es hier die Aussage $α\to β$ zu überprüfen.
        Wir verwenden dabei, die in den Aufgabenteilen (a), (b) und (c) gezeigten Aussagen.
        \begin{align*}
          &\mathscr{A}\fulfills α\to β \\
          &\iff \mathscr{A}\not\fulfills α \quad \text{oder} \quad \mathscr{A}\fulfills β \\
          &\implies \mathscr{A}_α\fregeProofable \lnot α \quad \text{oder} \quad \mathscr{A}_β\fregeProofable β \\
          &\implies \mathscr{A}_{α\toβ}\fregeProofable \lnot α \quad \text{oder} \quad \mathscr{A}_{α\toβ}\fregeProofable β \\
          &\stackrel{\text{(a),(b)}}{\implies}
            \begin{aligned}[t]
              &\mathscr{A}_{α\to β} \fregeProofable α\to β \quad \text{oder} \\
              &\mathscr{A}_{α\to β} \fregeProofable α\to β
            \end{aligned} \\
          &\implies \mathscr{A}_{α\to β} \fregeProofable α\to β
        \end{align*}
        \begin{align*}
          &\mathscr{A}\not\fulfills α\to β \\
          &\iff \mathscr{A} \fulfills α \quad \text{und} \quad \mathscr{A} \not\fulfills β \\
          &\implies \mathscr{A}_α\fregeProofable α \quad \text{und} \quad \mathscr{A}_β\fregeProofable \lnot β \\
          &\implies \mathscr{A}_{α\toβ}\fregeProofable α \quad \text{und} \quad \mathscr{A}_{α\toβ}\fregeProofable \lnot β \\
          &\stackrel{(c)}{\implies} \mathscr{A}_{α\to β} \fregeProofable \lnot(α\to β)
        \end{align*}
      % paragraph induktionsschluss (end)

      Damit sind die gewünschten Aussagen für beliebige Formeln α durch vollständige Induktion gezeigt worden. \qedBox
    % paragraph  (end)

  % section aufgabe_19 (end)

  \section*{Aufgabe 21} % (fold)
  \label{sec:aufgabe_21}

    Im Folgenden seien α, β und γ drei beliebige Formeln.
    Wir definieren zunächst die Axiome des Mendelson-Kalküls.
    \begin{alignat*}{3}
      &\mendelsonAxiomI(α,β)  &&\define\ α\to(β\to α) \\
      & &&=\ \fregeAxiomI(α,β) \\
      &\mendelsonAxiomII(α,β,γ)  &&\define\
      \begin{aligned}[t]
        &(α\to(β\to γ)) \\
        &\to ((α\to β)\to(α\to γ))
      \end{aligned}\\
      & &&=\ \fregeAxiomII(α,β,γ) \\
      &\mendelsonAxiomIII(α,β)  &&\define\ (\lnot β\to\lnot α)\to((\lnot β\to α)\to β)
    \end{alignat*}
    Die einzige definierte Schlussregel im Mendelson-Kalkül ist wieder \enquote{Modus Ponens}.
    Auch die Bergriffe der Herleitung und der Beweisbarkeit im Mendelson-Kalkül können analog zum Frege-Kalkül definiert werden.
    Wir schreiben $\mendelsonProofable φ$ für eine beliebige Mendelson-beweisbare Formel φ.

    Da die ersten beiden Axiome des Mendelson-Kalküls gleich den ersten beiden Axiomen des Frege-Kalküls sind und beide Kalküle dieselbe Schlussregel verwenden, gelten das Theorem $\mathrm{ID}(α)$ und die Schlussregeln $\mathrm{TT}(α,β)$ und $\mathrm{TRANS}(α\to β,β\to γ)$ auch im Mendelson-Kalkül für beliebige Formeln α,β und γ, da ihn ihren Herleitungen nur die ersten beiden Axiome zusammen mit \enquote{Modus Ponens} verwendet werden.

    Um nun die Korrektheit des Mendelson-Kalküls für beliebige Formeln φ zu beweisen, führen wir einen Induktionsbeweis über die Länge der Herleitung einer Formel im Mendelson-Kalkül.
    \paragraph{Induktionsanfang:} % (fold)
    \label{par:induktionsanfang}
      Es sei nun φ eine Formel, die innerhalb des Mendelson-Kalküls durch einen einzigen Schritt hergeleitet werden kann.
      Demzufolge muss es sich bei φ um ein Axiom des Mendelson-Kalküls handeln.
      Im folgenden sein α, β und γ beliebige Formeln.

      \paragraph{Fall $φ=\mendelsonAxiomI(α,β) = \fregeAxiomI(α,β)$} % (fold)
        Durch den Vollständigkeitssatz des Frege-Kalküls wurde bereits gezeigt, dass das erste Axiom des Frege-Kalküls für beliebige α und β gültig ist.
      % paragraph fall_ (end)
      \paragraph{Fall $φ=\mendelsonAxiomII(α,β,γ) = \fregeAxiomII(α,β,γ)$} % (fold)
        Durch den Vollständigkeitssatz des Frege-Kalküls wurde bereits gezeigt, dass das zweite Axiom des Frege-Kalküls für beliebige α, β und γ gültig ist.
      % paragraph fall_ (end)
      \paragraph{Fall $φ=\mendelsonAxiomIII(α,β)$} % (fold)
        Für das dritte Axiom des Mendelson-Kalküls lässt sich die folgende Äquivalenzkette aufstellen.
        \begin{align*}
          φ &= (\lnot β\to\lnot α)\to((\lnot β\to α)\to β) \\
          &\equiv \lnot(\lnot β\to\lnot α)\lor((\lnot β\to α)\to β) \\
          &\equiv (\lnot β\land\lnot\lnot α)\lor(\lnot(\lnot β \to α)\lor β) \\
          &\equiv (\lnot β\land α)\lor((\lnot β \land \lnot α)\lor β) \\
          &\equiv ((\lnot β\land α)\lor(\lnot β\land\lnot α))\lor β \\
          &\equiv (\lnot β \land (α\lor\lnot α))\lor β \\
          &\equiv (\lnot β \land \top) \lor β
          \equiv \lnot β \lor β
          \equiv \top
        \end{align*}
        Da φ äquivalent zu $\top$ ist, muss es sich bei φ um eine gültige Formel handeln und demzufolge $\fulfills φ$ gelten.
      % paragraph fall_ (end)

      Da α, β und γ beliebig gewählt wurden, sind damit die Axiome des Mendelson-Kalküls gültig.
      Als Folge ist damit auch jede Herleitung im Mendelson-Kalkül mit einem einzigen Schritt gültig.
    % paragraph induktionsanfang (end)
    \paragraph{Induktionsvoraussetzung:} % (fold)
    \label{par:induktionsvoraussetzung_}
      Es sei $n\in\setNatural$.
      Jede Formel φ, welche eine Herleitung im Mendelson-Kalkül mit einer Länge kleiner gleich $n$ besitzt, ist gültig.
    % paragraph induktionsvoraussetzung_ (end)
    \paragraph{Induktionsschluss:} % (fold)
    \label{par:induktionsschluss_}
      Sei nun φ eine Formel, für welche eine Herleitung $(α_k)_\set{k\in\setNatural}{k\leq n+1}$ im Mendelson-Kalkül mit der Länge $n+1$ existiert.
      \paragraph{Fall $α_{n+1}$ ist ein Axiom:} % (fold)
        Handelt es sich bei $α_{n+1}$ um ein Axiom, so ist aus dem Induktionsanfang ersichtlich, dass es sich bei $φ=α_{n+1}$ auch um eine gültige Formel handeln muss.
      % paragraph fall_ (end)
      \paragraph{Fall $α_{n+1}$ ist kein Axiom:} % (fold)
        Handelt es sich bei $α_{n+1}$ nicht um ein Axiom, so kann $φ=α_{n+1}$ nur aus der Schlussregel $\modusPonens(α_i,α_j)$ mit $i,j\in\setNatural$ und $i,j\leq n$ entstehen, wobei $α_j = α_i\to φ$ gilt.
        Nach der Induktionsvoraussetzung handelt es sich bei $α_i$ und $α_j$ um gültige Formeln, da für sie Herleitungen im Mendelson-Kalkül der Länge $i$ beziehungsweise $j$ existieren.
        Des Weiteren gilt nach einem bereits bewiesenen Satz das Folgende.
        \[
          \fulfills α_i \quad \text{und}\quad \fulfills α_i\toφ \implies \fulfills φ
        \]
        Es wurde damit gezeigt, dass es sich bei φ wieder um eine gültige Formel handelt.
      % paragraph fall_ (end)
    % paragraph induktionsschluss_ (end)

    Die Induktion zeigt, dass jede Formel φ, für welche eine Herleitung im Mendelson-Kalkül existiert, auch gültig ist.
    Das Mendelson-Kalkül ist damit korrekt.
    \[
      \mendelsonProofable φ \implies \fulfills φ
    \]

    Für die Vollständigkeit des Mendelson-Kalküls verwenden wir den Vollständigkeitssatz des Frege-Kalküls.
    Dafür beweisen wir zunächst, dass jede Frege-beweisbare Formel φ auch Mendelson-beweisbar ist, indem wir eine Induktion über die Länge der Herleitung von φ im Frege-Kalkül durchführen.
    \[
      \fregeProofable φ \implies \mendelsonProofable φ
    \]
    \paragraph{Induktionsanfang:} % (fold)
    \label{par:induktionsanfang_}
      Handelt es sich bei φ um eine Formel, für die eine Herleitung im Frege-Kalkül mit einem einzigen Schritt existiert, so muss es sich bei φ um ein Axiom des Frege-Kalküls handeln.
      Offensichtlich sind die ersten beiden Axiome des Frege-Kalküls Mendelson-beweisbar durch die ersten beiden Axiome des Mendelson-Kalküls, da diese gleich sind.
      Im Folgenden sei α eine beliebige Formel.
      Der noch zu betrachtende Fall lautet $φ=\fregeAxiomIII(α)$.
      Die Mendelson-Beweisbarkeit von φ lässt sich in diesem Falle durch den folgenden direkten Beweis zeigen.
      \begin{align}
        \tag{1}
          & \perp\to\perp
          && \mathrm{ID}(\perp) \\
        \tag{2}
          & (α\to\perp)\to(\perp\to\perp)
          && \mathrm{TT}((1),α\to\perp) \\
        \tag{3}
          &
            \begin{aligned}[t]
              &(\lnot α\to \lnot\perp) \\
              &\to ((\lnot α\to\perp)\to α)
            \end{aligned}
          && \mendelsonAxiomIII(\perp,α) \\
        \tag{4}
          & \lnot α \to\lnot\perp
          && (2)\quad
            \begin{aligned}[c]
              &\lnot α = α\to\perp \\
              &\lnot\perp = \perp\to\perp
            \end{aligned} \\
        \tag{5}
          & ((\lnot α\to\perp)\to α)
          && \modusPonens((4),(3)) \\
        \tag{6}
          & \lnot\lnot α \to α
          && (5), \lnot\lnot α = \lnotα\to\perp
      \end{align}
      Damit ist jede Frege-beweisbare Formel φ mit einer Herleitung im Frege-Kalkül der Länge Eins auch Mendelson-beweisbar.
    % paragraph induktionsanfang_ (end)
    \paragraph{Induktionsvoraussetzung:} % (fold)
    \label{par:induktionsvoraussetzung}
      Sei $n\in\setNatural$.
      Jede Frege-beweisbare Formel φ mit einer Herleitung im Frege-Kalkül der Länge kleiner gleich $n$ ist auch Mendelson-beweisbar.
    % paragraph induktionsvoraussetzung (end)
    \paragraph{Induktionsschluss:} % (fold)
    \label{par:induktionsschluss_}
      Es sei nun φ eine Frege-beweisbare Formel mit Herleitung $(α_k)_\set{k\in\setNatural}{k\leq n+1}$ im Frege-Kalkül der Länge $n+1$.
      \paragraph{Fall $α_{n+1}$ ist ein Axiom:} % (fold)
        Handelt es sich bei $α_{n+1}$ um ein Axiom, so ist aus dem Induktionsanfang ersichtlich, dass es sich bei $φ=α_{n+1}$ auch um eine Mendelson-beweisbare Formel handeln muss.
      % paragraph fall_ (end)
      \paragraph{Fall $α_{n+1}$ ist kein Axiom:} % (fold)
        Handelt es sich bei $α_{n+1}$ nicht um ein Axiom, so kann $φ=α_{n+1}$ nur aus der Schlussregel $\modusPonens(α_i,α_j)$ mit $i,j\in\setNatural$ und $i,j\leq n$ entstehen, wobei $α_j = α_i\to φ$ gilt.
        Nach der Induktionsvoraussetzung handelt es sich bei $α_i$ und $α_j$ aber auch um Mendelson-beweisbare Formeln, da für sie Herleitungen im Frege-Kalkül der Länge $i$ beziehungsweise $j$ existieren.
        Nach $\modusPonens(α_i,α_j)$ im Mendelson-Kalkül ist auch $φ$ Mendelson-beweisbar.
        Es wurde damit gezeigt, dass es sich bei φ wieder um eine Mendelson-beweisbare Formel handelt.
    % paragraph induktionsschluss_ (end)

    Die Induktion zeigt, dass jede Frege-beweisbare Formel φ auch Mendelson-beweisbar ist.
    \[
      \fregeProofable φ \implies \mendelsonProofable φ
    \]
    Der Vollständigkeitssatz des Frege-Kalküls besagt das Folgende für beliebige Formeln φ.
    \[
      \fregeProofable φ \iff \fulfills φ
    \]
    Demzufolge ist jede gültige Formel φ auch Mendelson-beweisbar.
    Damit ist das Mendelson-Kalkül vollständig.
    \[
      \fulfills φ \implies \mendelsonProofable φ
    \]
    Durch die Anwendung der zuvor gezeigten Korrektheit des Mendelson-Kalküls ist die folgende Aussage für beliebige Formeln φ ersichtlich.
    \[
      \fulfills φ \iff \mendelsonProofable φ
    \]
    Der Vollständigkeitssatz gilt demnach auch für das Mendelson-Kalkül. \qedBox

  % section aufgabe_21 (end)

  \end{multicols}

\end{document}