\documentclass[9pt,fleqn,twoside,a4paper]{article}
\usepackage{extsizes}
\usepackage[a4paper,left=27mm,right=27mm,top=30mm,bottom=30mm]{geometry}
\linespread{1.15}\selectfont
\usepackage[utf8]{inputenc}
\usepackage[T1]{fontenc}
\usepackage[ngerman]{babel}
\usepackage{times}
\usepackage[leqno]{amsmath,mathtools}
\allowdisplaybreaks
\usepackage{amssymb}
\usepackage{dsfont}
\usepackage[mathscr]{euscript}
\usepackage{multicol}
\usepackage[autostyle,german=guillemets]{csquotes}
\usepackage{turnstile}
\usepackage{enumitem}

\usepackage{fancyhdr}
\fancypagestyle{titlestyle}{
  \fancyhf{}
  % \fancyfoot[C]{\footnotesize\bigskip\thepage/\pageref{LastPage}}
  \fancyfoot[C]{\footnotesize\bigskip\thepage}
  \renewcommand{\footrulewidth}{0.5pt}
  \renewcommand{\headrulewidth}{0pt}
}
\fancypagestyle{mainstyle}{
  \fancyhf{}
  \fancyfoot[C]{\footnotesize\bigskip\thepage}
  \fancyhead[LO,RE]{\footnotesize \leftmark \smallskip} %left
  \fancyhead[RO,LE]{\footnotesize \rightmark \smallskip} %right
  \renewcommand{\footrulewidth}{0.5pt}
  \renewcommand{\headrulewidth}{0.5pt}
}
% \pagestyle{mainstyle}

\usepackage{uniinput}
\usepackage{utilities}

\renewcommand{\separate}{\quad}
\renewcommand{\implies}{\quad\Longrightarrow\quad}
\renewcommand{\iff}{\quad\Longleftrightarrow\quad}
\newcommand{\fregeProofable}{\sststile{\mathrm{Fre}}{}{}}
\newcommand{\fregeAxiomI}{\mathrm{F1}}
\newcommand{\fregeAxiomII}{\mathrm{F2}}
\newcommand{\fregeAxiomIII}{\mathrm{F3}}
\newcommand{\mendelsonProofable}{\sststile{\mathrm{Men}}{}{}}
\newcommand{\mendelsonAxiomI}{\mathrm{M1}}
\newcommand{\mendelsonAxiomII}{\mathrm{M2}}
\newcommand{\mendelsonAxiomIII}{\mathrm{M3}}
\newcommand{\modusPonens}{\mathrm{MP}}
\newcommand{\fulfills}{\sdtstile{}{}{}}
\newcommand{\infers}{\ddtstile{}{}{}}
\newcommand{\qedBox}{\hfill\ensuremath{\square}}

\begin{document}
  \pagestyle{mainstyle}
  \thispagestyle{titlestyle}
  \hrule
  \begin{center}
    \huge
    % \bfseries
    \scshape
    Logiksysteme \\ Übungsserie 9: Lösungen
  \end{center}
  \medskip
  \footnotesize
  \begin{minipage}[c]{0.49\textwidth}
    Markus Pawellek \\
    markuspawellek@gmail.com
  \end{minipage}
  \hfill
  \begin{minipage}[c]{0.49\textwidth}
    \raggedleft
    \today
  \end{minipage}
  \medskip
  \normalsize
  \hrule
  \bigskip

  \begin{multicols}{2}

  \section*{Aufgabe 35} % (fold)
  \label{sec:aufgabe_35}

    Es seien $X\neq\emptyset$ eine nichtleere Menge und $R\subset X\times X$ eine binäre Relation auf $X$.
    Wir wollen zeigen, dass die beiden folgenden Aussagen äquivalent sind.
    \begin{enumerate}[label=\text{(\roman*)}]
      \item{
        \label{1}
        $R$ ist reflexiv, transitiv und symmetrisch.
      }
      \item{
        \label{2}
        $R$ ist reflexiv und euklidisch.
      }
    \end{enumerate}
    \paragraph{$\ref{1}\Rightarrow\ref{2}$:} % (fold)
      Unter Annahme von \ref{1} wissen wir bereits, dass $R$ reflexiv ist.
      Es reicht also zu zeigen, dass $R$ euklidisch ist.
      Im Folgenden seien $x,y,z\in X$ beliebig gewählt.
      \begin{align*}
        &(x,y)\in R \quad\&\quad (x,z)\in R \\
        &\stackrel{\mathclap{\text{(Symmetrie)}}}{\implies} (y,x)\in R \quad\&\quad (x,z)\in R \\
        &\stackrel{\mathclap{\text{(Transitivität)}}}{\implies} (y,z)\in R
      \end{align*}
      Damit wurde gezeigt, dass $R$ reflexiv ist.
    % paragraph 1rightarrow2 (end)
    \paragraph{$\ref{2}\Rightarrow\ref{1}$:} % (fold)
      Unter Annahme von \ref{1} wissen wir bereits, dass $R$ reflexiv ist.
      Es reicht also zu zeigen, dass $R$ transitiv und symmetrisch ist.
      Im Folgenden seien wieder $x,y,z\in X$ beliebig gewählt.
      Wir wollen zuerst die Symmetrie zeigen.
      \begin{align*}
        &(x,y)\in R \\
        &\stackrel{\mathclap{\text{(Reflexivität)}}}{\implies} (x,y)\in R \quad\&\quad (x,x)\in R \\
        &\stackrel{\mathclap{\text{(Euklidizität)}}}{\implies} (y,x)\in R
      \end{align*}
      Durch Verwendung der Symmetrie lässt sich nun auch die Transitivität zeigen.
      \begin{align*}
        &(x,y)\in R \quad\&\quad (y,z)\in R \\
        &\stackrel{\mathclap{\text{(Symmetrie)}}}{\implies} (y,x)\in R \quad\&\quad (y,z)\in R \\
        &\stackrel{\mathclap{\text{(Euklidizität)}}}{\implies} (x,z)\in R
      \end{align*}
    % paragraph paragraph_name (end)

    In allen Fällen waren die gewählten Variablen beliebig.
    Demnach wurde die gewünschte Äquivalenz gezeigt. \qedBox
  % section aufgabe_35 (end)

  \end{multicols}

\end{document}