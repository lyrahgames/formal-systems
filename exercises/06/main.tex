\input{sty/pre}

\title{Logiksysteme \\ Übungsserie 6}
\author{Markus Pawellek}
% \date{}
\newcommand{\email}{markuspawellek@gmail.com}

\usepackage{multicol}
\usepackage{turnstile}
\usepackage{sty/uniinput}

\usepackage {tikz}
\usetikzlibrary {positioning}
\usetikzlibrary{arrows.meta}
%\usepackage {xcolor}
\definecolor{processblue}{cmyk}{1,1,1,0}

\begin{document}

	\articletitle

  \begin{multicols}{2}

    \newcommand{\kripke}{\sdtstile{\m{K}}{}{}}
    \newcommand{\lbox}{\Box}
    \newcommand{\ldiamond}{\Diamond}
    \newcommand{\ctrue}{\text{wahr}}

    \newenvironment{graph}{
      \begin{center}
      \begin{tikzpicture}
      [
        -latex, auto,
        node distance =0.1\textwidth and 0.1\textwidth,
        on grid, semithick,
        state/.style = {
          circle,
          top color = white,
          bottom color = processblue!10,
          draw,
          processblue,
          text=black,
          minimum width = 0.7 cm,
        },
        >=Stealth
      ]
    }{
      \end{tikzpicture}
      \end{center}
    }


    \section*{Aufgabe 22} % (fold)
    \label{sec:aufgabe_22}

      \paragraph{(1):}
      Für die Konstruktion eines erfüllenden Kripke-Modells, verwenden wir zunächst die folgende Äquivalenzkette.
      \[
        \begin{aligned}
          &(A\land\lbox(A\to B)) \to \lbox B \\
          &\quad\equiv \lnot(A\land\lbox(A\to B)) \lor \lbox B \\
          &\quad\equiv (\lnot A \lor \ldiamond(A\land\lnot B)) \lor \lbox B
        \end{aligned}
      \]
      Es folgt unmittelbar, dass es sich bei dem folgenden Kripke-Modell mit Startwelt $s$, um ein erfüllendes Modell handelt.
      \begin{graph}
        \node[state,label=$s$] (A) {};
      \end{graph}
      Für die Konstruktion eines nicht-erfüllenden Kripke-Modells, betrachten wir analog die negierte Aussage.
      \[
        \begin{aligned}
          &\lnot((A\land\lbox(A\to B)) \to \lbox B) \\
          &\quad\equiv (A\land\lbox(A\to B)) \land \lnot\lbox B \\
          &\quad\equiv (A\land\lbox(\lnot A\lor B)) \land \ldiamond\lnot B
        \end{aligned}
      \]
      Demzufolge erfüllt das folgende Modell mit Startwelt $s$ die negierte Formel, was wiederum äquivalent dazu ist, dass die ursprüngliche Formel von diesem Modell mit Startwelt $s$ nicht erfüllt wird.
      \begin{graph}
        \node[state,label=$s$] (A) {$A$};
        \node[state] (B) [right=of A] {};
        \path (A) edge node {} (B);
      \end{graph}

      \paragraph{(2):}
      Zunächst vereinfachen wir wieder die gegebene Formel, um das erfüllende Modell zu konstruieren.
      \[
        \begin{aligned}
          \Diamond A\to\Box A
          \equiv \lnot\ldiamond A \lor \lbox A
          \equiv \lbox\lnot A \lor \lbox A
        \end{aligned}
      \]
      \begin{graph}
        \node[state,label=$s$] (A) {};
      \end{graph}
      Jetzt betrachten wir die Negation, um das nicht-erfüllende Modell zu konstruieren.
      \[
        \begin{aligned}
          \lnot(\ldiamond A \to \lbox A) \equiv \ldiamond A \land \ldiamond\lnot A
        \end{aligned}
      \]
      \begin{graph}
        \node[state,label=$s$] (A) {};
        \node[state] (B) [right = of A] {$A$};
        \path (A) edge [loop left] node [left] {} (A);
        \path (A) edge [] node [] {} (B);
      \end{graph}

      \paragraph{(3):} % (fold)
      Zunächst vereinfachen wir wieder die gegebene Formel, um das erfüllende Modell zu konstruieren.
      \[
        \ldiamond\lbox A \to \lbox\ldiamond A \equiv \lnot\ldiamond\lbox A \lor \lbox\ldiamond A
      \]
      \begin{graph}
        \node[state,label=$s$] (S) {};
      \end{graph}
      Jetzt betrachten wir die Negation, um das nicht-erfüllende Modell zu konstruieren.
      \[
        \begin{aligned}
          \lnot(\ldiamond\lbox A \to \lbox\ldiamond A) \equiv \ldiamond\lbox A \land \ldiamond\lbox\lnot A
        \end{aligned}
      \]
      \begin{graph}
        \node[state,label=$s$] (S) {};
        \node[state] (A) [right = of S] {};
        \node[state] (B) [right = of A] {$A$};
        \path (S) edge [loop left] node {} (S);
        \path (S) edge [] node {} (A);
        \path (A) edge [] node {} (B);
      \end{graph}

      \paragraph{(4):}
      Die gegebene Formel liegt bereits in einer vereinfachten Form vor.
      Das erfüllende Modell kann also direkt abgelesen werden.
      \[
        \lbox\lbox A \land \ldiamond\ldiamond B
      \]
      % \begin{graph}
      %   \node[state,label=$s$] (S) {};
      %   \node[state] (A) [right = of S] {};
      %   \node[state] (B) [right = of A] {$A,B$};
      %   \path (S) edge [] node {} (A);
      %   \path (A) edge [] node {} (B);
      % \end{graph}
      \begin{graph}
        \node[state,label=$s$] (S) {$A,B$};
        \node[state] (A) [right = of S] {};
        \path (S) edge [bend left = 20] node {} (A);
        \path (A) edge [bend left = 20] node {} (S);
      \end{graph}
      Jetzt betrachten wir die Negation, um das nicht-erfüllende Modell zu konstruieren.
      \[
        \lnot(\lbox\lbox A \land \ldiamond\ldiamond B) \equiv \ldiamond\ldiamond\lnot A \lor \lbox\lbox\lnot B
      \]
      % \begin{graph}
      %   \node[state,label=$s$] (S) {};
      %   \node[state] (A) [right = of S] {};
      %   \node[state] (B) [right = of A] {$B$};
      %   \path (S) edge [] node {} (A);
      %   \path (A) edge [] node {} (B);
      % \end{graph}
      \begin{graph}
        \node[state,label=$s$] (S) {};
        \node[state] (A) [right = of S] {};
        \path (S) edge [bend left = 20] node {} (A);
        \path (A) edge [bend left = 20] node {} (S);
      \end{graph}

      \paragraph{(5):}
      Zunächst vereinfachen wir wieder die gegebene Formel, um das erfüllende Modell zu konstruieren.
      \[
        \begin{aligned}
          &(\ldiamond(A\to B)\land\ldiamond(B\to C)) \to \ldiamond(A\to C) \\
          &\quad\equiv \lnot(\ldiamond(A\to B)\land\ldiamond(B\to C)) \lor \ldiamond(A\to C) \\
          &\quad\equiv (\lbox(A\land\lnot B)\lor\lbox(B\land\lnot C)) \lor \ldiamond(\lnot A \lor C)
        \end{aligned}
      \]
      \begin{graph}
        \node[state,label=$s$] (S) {};
        \node[state] (A) [right = of S] {$C$};
        \path (S) edge [] node {} (A);
      \end{graph}
      Jetzt betrachten wir die Negation, um das nicht-erfüllende Modell zu konstruieren.
      \[
        \begin{aligned}
          &\lnot((\ldiamond(A\to B)\land\ldiamond(B\to C)) \to \ldiamond(A\to C)) \\
          &\equiv (\ldiamond(A\to B)\land\ldiamond(B\to C)) \land \lbox(A\land\lnot C)
        \end{aligned}
      \]
      % \begin{graph}
      %   \node[state,label=$s$] (S) {};
      %   \node[state] (A) [right = of S] {$A,B$};
      %   \node[state] (B) [below right = of S] {$A$};
      %   \path (S) edge [] node {} (A);
      %   \path (S) edge [] node {} (B);
      % \end{graph}
      \begin{graph}
        \node[state,label=$s$] (S) {$A,B$};
        \node[state] (B) [right = of S] {$A$};
        \path (S) edge [loop left] node {} (S);
        \path (S) edge [] node {} (B);
      \end{graph}

      \paragraph{(6):}
      Die gegebene Formel liegt bereits in einer vereinfachten Form vor.
      Das erfüllende Modell kann also direkt abgelesen werden.
      \[
        \ldiamond(A\land\lbox(\lnot A\land\ldiamond A))
      \]
      \begin{graph}
        \node[state,label=$s$] (S) {};
        \node[state] (A) [right = of S] {$A$};
        \path (S) edge [] node {} (A);
      \end{graph}
      Jetzt betrachten wir die Negation, um das nicht-erfüllende Modell zu konstruieren.
      \[
        \begin{aligned}
          &\lnot(\ldiamond(A\land\lbox(\lnot A\land\ldiamond A))) \\
          &\quad\equiv \lbox(\lnot A \lor \ldiamond(A\lor \lbox\lnot A))
        \end{aligned}
      \]
      \begin{graph}
        \node[state,label=$s$] (S) {};
      \end{graph}

      \paragraph{(7):}
      Zunächst vereinfachen wir wieder die gegebene Formel, um das erfüllende Modell zu konstruieren.
      \[
        \begin{aligned}
          &(\ldiamond A \land \ldiamond B)\to \ldiamond(A\land B) \\
          &\quad\equiv \lnot(\ldiamond A \land \ldiamond B) \lor \ldiamond(A\land B)
        \end{aligned}
      \]
      \begin{graph}
        \node[state,label=$s$] (S) {};
        \node[state] (A) [right = of S] {$A,B$};
        \path (S) edge [] node {} (A);
      \end{graph}
      Jetzt betrachten wir die Negation, um das nicht-erfüllende Modell zu konstruieren.
      \[
        \begin{aligned}
          &\lnot((\ldiamond A \land \ldiamond B)\to \ldiamond(A\land B)) \\
          &\quad\equiv (\ldiamond A \land \ldiamond B) \land \lbox(\lnot A \lor \lnot B)
        \end{aligned}
      \]
      % \begin{graph}
      %   \node[state,label=$s$] (S) {};
      %   \node[state] (A) [right = of S] {$A$};
      %   \node[state] (B) [below right = of S] {$B$};
      %   \path (S) edge [] node {} (A);
      %   \path (S) edge [] node {} (B);
      % \end{graph}
      \begin{graph}
        \node[state,label=$s$] (S) {$B$};
        \node[state] (A) [right = of S] {$A$};
        \path (S) edge [] node {} (A);
        \path (S) edge [loop left] node {} (S);
      \end{graph}

      \paragraph{(8):}
      Zunächst vereinfachen wir wieder die gegebene Formel, um das erfüllende Modell zu konstruieren.
      \[
        \begin{aligned}
          &(A\to\lbox A)\to(\lbox A \to \lbox\lbox A) \\
          &\quad\equiv \lnot(A\to\lbox A) \lor (\lbox A \to \lbox\lbox A) \\
          &\quad\equiv (A\land\ldiamond\lnot A) \lor (\ldiamond\lnot A \lor \lbox\lbox A)
        \end{aligned}
      \]
      \begin{graph}
        \node[state,label=$s$] (S) {};
      \end{graph}
      Jetzt betrachten wir die Negation, um das nicht-erfüllende Modell zu konstruieren.
      \[
        \begin{aligned}
          &\lnot((A\to\lbox A)\to(\lbox A \to \lbox\lbox A)) \\
          &\quad\equiv (\lnot A\lor\lbox A) \land (\lbox A \land \ldiamond\ldiamond\lnot A)
        \end{aligned}
      \]
      \begin{graph}
        \node[state,label=$s$] (S) {};
        \node[state] (A) [right = of S] {$A$};
        \node[state] (B) [right = of A] {};
        \path (S) edge [] node {} (A);
        \path (A) edge [] node {} (B);
      \end{graph}

      \paragraph{(9):}
      Die gegebene Formel liegt bereits in einer vereinfachten Form vor.
      Das erfüllende Modell kann also direkt abgelesen werden.
      \[
        \ldiamond A \land \ldiamond\lnot A \land \lbox(((A\land\lbox A)\lor(\lnot A\land \lbox\lnot A)) \land\ldiamond B\land \ldiamond\lnot B)
      \]
      % \begin{graph}
      %   \node[state,label=$s$] (S) {};
      %   \node[state] (A) [right = of S] {$A$};
      %   \node[state] (B) [right = of A] {$A,B$};
      %   \node[state] (C) [below = of S] {};
      %   \node[state] (D) [right = of C] {B};
      %   \path (S) edge [] node {} (A);
      %   \path (A) edge [] node {} (B);
      %   \path (S) edge [] node {} (C);
      %   \path (C) edge [] node {} (D);
      %   \path (A) edge [loop below] node {} (A);
      %   \path (C) edge [loop left] node {} (C);
      % \end{graph}
      \begin{graph}
        \node[state,label=$s$] (S) {$A,B$};
        \node[state] (A) [right = of S] {$A$};
        % \node[state] (B) [right = of A] {$A,B$};
        \node[state] (C) [below = of S] {};
        \node[state] (D) [right = of C] {B};
        \path (S) edge [bend right] node {} (A);
        \path (A) edge [bend right] node {} (S);
        \path (S) edge [] node {} (C);
        \path (C) edge [] node {} (D);
        \path (A) edge [loop below] node {} (A);
        \path (C) edge [loop left] node {} (C);
      \end{graph}
      Eine Betrachtung der Negation ist hier nicht nötig, da es für die Nichterfüllung ausreichend ist, ein Modell zu konstruieren, dessen Startwelt keine Nachfolger hat.
      \begin{graph}
        \node[state,label=$s$] (S) {};
      \end{graph}

    % section aufgabe_22 (end)

    \section*{Aufgabe 23} % (fold)
    \label{sec:aufgabe_23}

      Eine beliebige modal-logische Formel φ ist nach Definition genau dann gültig, wenn für alle Kripke-Modelle \e{M} und alle Welten $w$ von $\e{M}$ gilt, dass $\e{M},w\kripke φ$.
      Nach Definition gilt $φ\equiv\top$ genau dann, wenn für alle Kripke-Modelle \e{M} und alle Welten $w$ von \e{M} das Folgende gilt.
      \[
        \e{M},w\kripke φ \iff \e{M},w\kripke\top \stackrel{\text{def.}}{\iff} \ctrue
      \]
      Ist φ also gültig, so muss auch $φ\equiv\top$ gelten.
      Gilt $φ\equiv\top$, muss φ wiederum gültig sein.
      \[
        φ \text{ ist gültig} \iff φ\equiv\top
      \]
      Für die Aufgaben reicht es demnach die Äquivalenz zu $\top$ zu beweisen.
      Im Folgenden verwende ich vor allem die De Morganschen Gesetze, die aus dem Frege-Kalkül bekannten Äquivalenzen und die Äquivalenzen des Lemmas 6.6.

      \paragraph{(1):} % (fold)
      Die Gültigkeit der Formel kann mithilfe einer einfachen Äquivalenzkette gezeigt werden.
      \begin{align*}
        \Box(&A→B) → (\Diamond A→\Diamond B) \\
        &\equiv ¬\Box(A\to B) \lor (\Diamond A \to \Diamond B) \\
        &\equiv \Diamond(A\land \lnot B) \lor (\lnot\Diamond A \lor \Diamond B) \\
        &\equiv \Diamond(A\land\lnot B) \lor (\lnot\Diamond A \lor \Diamond B) \\
        &\equiv (\Diamond(A\land\lnot B) \lor \Diamond B) \lor \lnot\Diamond A \\
        &\equiv \Diamond((A\land\lnot B)\lor B) \lor \lnot\Diamond A \\
        &\equiv \Diamond((A\lor B) \land (\lnot B \lor B)) \lor \lnot\Diamond A \\
        &\equiv \Diamond(A\lor B) \lor \lnot\Diamond A \\
        &\equiv (\Diamond A \lor \Diamond B) \lor \lnot\Diamond A \\
        &\equiv \Diamond B \lor (\Diamond A \lor \lnot\Diamond A) \equiv \Diamond B \lor \top \equiv \top
      \end{align*}
      Die Gültigkeit wurde damit gezeigt.\qedbox

      \paragraph{(2):}
      Der Einfachheit halber betrachten wir zunächst nur den ersten Teil der Formel.
      \begin{align*}
        \Box A&\land \Box(A\to B) \\
        &\equiv \Box(A\land(A\to B)) \equiv \Box(A\land(\lnot A \lor B)) \\
        &\equiv \Box(A\land B) \equiv \Box A \land \Box B
      \end{align*}
      Diese \enquote{Nebenrechnung} verwenden wir nun für die eigentliche Äquivalenzkette.
      \begin{align*}
        (\Box A&\land \Box(A\to B)) \to \Box B \\
        &\equiv \lnot(\Box A\land \Box(A\to B)) \lor \Box B \\
        &\equiv \lnot (\Box A \land \Box B) \lor \Box B \\
        &\equiv (\lnot\Box A \lor \lnot\Box B) \lor \Box B \\
        &\equiv \lnot\Box A \lor (\lnot\Box B \lor \Box B) \\
        &\equiv \lnot\Box A \lor \top \equiv \top
      \end{align*}
      Die Gültigkeit der Formel wurde damit gezeigt.\qedbox

      \paragraph{(3):}
      Bei dieser Formel eignet es sich in einer hierarchischen Struktur vorzugehen und zunächst einfachere Subformeln zu betrachten.
      \begin{align*}
        A &\equiv A \land \top \equiv A \land (B\lor\lnot B) \\
        &\equiv (A\land B) \lor (A\land\lnot B)
      \end{align*}
      Dieses Resultat soll nun verwendet werden, um eine alternative äquivalente Formel für die rechte Seite der gegebenen Formel zu bestimmen.
      \[\begin{aligned}
        \Diamond A &\equiv \Diamond((A\land B) \lor (A\land\lnot B)) \\
        &\equiv \Diamond(A\land B) \lor \Diamond(A\land\lnot B)
      \end{aligned}\]
      \[\begin{aligned}
        \Diamond B &\equiv \Diamond(A\land B) \lor \Diamond(\lnot A\land B)
      \end{aligned}\]
      \[\begin{aligned}
        \Diamond A &\land \Diamond B \\
        &\equiv
        \begin{aligned}[t]
          &(\Diamond(A\land B) \lor \Diamond(A\land\lnot B)) \\
          &\land (\Diamond(A\land B) \lor \Diamond(\lnot A\land B))
        \end{aligned} \\
        &\equiv \Diamond(A\land B) \lor (\Diamond(A\land\lnot B)\land\Diamond(\lnot A\land B))
      \end{aligned}\]
      Durch die gezeigten Äquivalenzen lässt sich nun das Folgende notieren.
      \begin{align*}
        &\Diamond(A\land B)\to(\Diamond A\land \Diamond B) \\
        &\equiv \lnot\Diamond(A\land B) \lor (\Diamond A\land\Diamond B) \\
        &\equiv
        \begin{aligned}[t]
          &\lnot\Diamond(A\land B) \\
          &\lor (\Diamond(A\land B) \lor (\Diamond(A\land\lnot B) \land\Diamond(\lnot A\land B)))
        \end{aligned} \\
        &\equiv
        \begin{aligned}[t]
          &(\lnot\Diamond(A\land B) \lor \Diamond(A\land B)) \\
          &\lor(\Diamond(A\land\lnot B) \land\Diamond(\lnot A\land B))
        \end{aligned} \\
        &\equiv \top \lor (\Diamond(A\land\lnot B) \land\Diamond(\lnot A\land B)) \equiv \top
      \end{align*}
      Die Gültigkeit der Formel wurde damit gezeigt. \qedbox

      \newcommand{\lequiv}{\leftrightarrow}

      \paragraph{(4):}
      Durch Verwendung der folgenden aus Lemma 6.6 bekannten Äquivalenz kann die Gültigkeit der Formel direkt gezeigt werden.
      \[
        \Diamond(A\to B) \equiv \Box A \to\Diamond B
      \]
      Wir verwenden zudem $φ\lequiv φ\equiv \top$ für alle modal-logischen Formeln φ.
      Es folgt nun die folgende Kette.
      \begin{align*}
        &\Diamond(A\to B) \lequiv (\Box A \to \Diamond B) \\
        &\equiv (\Box A \to \Diamond B) \lequiv (\Box A \to \Diamond B) \equiv \top
      \end{align*}
      Die Gültigkeit dieser Formel ist gezeigt. \qedbox

    % section aufgabe_23 (end)

    \section*{Aufgabe 24} % (fold)
    \label{sec:aufgabe_24}

      Es sei φ eine beliebige modal-logische Formel.
      Dann ist nach Definition $\lbox\ldiamond φ$ genau dann nicht gültig, wenn es ein Kripke-Modell $\e{M}$ mit Startwelt $s$ gibt, sodass $\e{M},s\not\kripke \lbox\ldiamond φ$ gilt.
      Dies ist aber gerade äquivalent dazu, dass es ein Kripke-Modell $\e{M}$ mit Startwelt $s$ gibt, sodass $\e{M},s\kripke\lnot\lbox\ldiamond φ$ gilt.
      Wir wissen nun das Folgende.
      \[
        \lnot\lbox\ldiamond φ \equiv \ldiamond\lbox\lnot φ
      \]
      Damit erfüllt das folgende Modell, welches nicht von φ abhängt, mit Startwelt $s$ die Formel $\lnot\lbox\ldiamond φ$.
      \begin{graph}
        \node[state,label=$s$] (S) {};
        \node[state] (A) [right = of S] {};
        \path (S) edge [] node [] {} (A);
      \end{graph}
      Insbesondere gibt es also ein Kripke-Modell mit ausgewählter Startwelt, welches $\lnot\lbox\ldiamond φ$ erfüllt und damit $\lbox\ldiamond φ$ nicht erfüllt.
      $\lbox\ldiamond φ$ ist demzufolge nicht gültig.
      Da φ beliebig gewählt wurde, ist damit die Aussage gezeigt. \qedbox

    % section aufgabe_24 (end)

  \end{multicols}

\end{document}